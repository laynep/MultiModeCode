\documentclass[11pt]{article}

\usepackage{amssymb}
\usepackage{amsmath}
\usepackage{fullpage}
\usepackage{color}

\newcommand{\mpl}{m_{\mathrm{pl}}}
\newcommand{\ud}{\mathrm{d}}
\newcommand{\pd}{\partial}
\def\calD{{\cal D}}
\def\bk{{\mathbf{k}}}
\def\f {{\phi}}

\setlength{\parskip}{1.5ex}
\setlength{\parindent}{0pt}
\def\tilC{{\tilde C}}
\def\calR{{\cal R}}
\def\calP{{\cal P}}
\def\calS{{\cal S}}
\def\calO{{\cal O}}
\def\df {{\dot \phi_0}}

\def\PIJ{{\mathcal{P}_{\delta \phi}^{IJ}}}
\def\Pten{{\mathcal P_h}}
\def\Pent{{\mathcal P_\mathrm{ent}}}
\def\Ppnad{{\mathcal P_\mathrm{Pnad}}}
\def\Cross{{\mathcal C_\mathrm{\calR \calS}}}
\def\PR{{\mathcal P_\calR}}
\def\PS{{\mathcal P_\calS}}

%Author comments.
%===============
\newcommand{\re}[1]{\textcolor{green}{[{\bf RE}: #1]}}
\newcommand{\lp}[1]{\textcolor{red}{[{\bf LP}: #1]}}
\newcommand{\jf}[1]{\textcolor{red}{[{\bf JF}: #1]}}
\newcommand{\jx}[1]{\textcolor{red}{[{\bf JX}: #1]}}
\newcommand{\hp}[1]{\textcolor{red}{[{\bf HP}: #1]}}

\title{\textsc{MultiModeCode}: \\ multifield perturbation solver}
\author{Richard Easther, Jonathan Frazer, Hiranya V. Peiris, Layne C. Price, and Jiajun Xu}
\date{\today}

\begin{document}

\maketitle

We start with the action for $N$ scalar fields $\phi^a$, $(a=1,2, \dots, N)$. 
\begin{equation}
S = \int \ud^4 x \sqrt{-g} \left[ -\frac{1}{2} \gamma_{ab} \pd_\mu \phi^a \pd^\mu \phi^b - V(\f^a) \right]
\end{equation}
We will first write down equations for the general metric $\gamma_{ab}$, and then reduce to the canonical case $\gamma_{ab} =\delta_{ab}$. 

The current stage of Multifield Modecode only deals with canonical fields. 

\section{The Classical Background}

\begin{equation}\label{bg_phi}
\ddot{\f^a} + \Gamma^a_{bc} \dot{\f^b} \dot{\f^c} + 3H \dot\f^a + \gamma^{ab} \frac{\pd V}{\pd \f^b} = 0  ~.
\end{equation}
Introducing the covariant derivative \lp{Need to clarify what this structure is... differential?}
\begin{equation}
D \f^a = \ud \f^a + \Gamma^a_{bc}\dot{\f^b}\ud\f^c ~,
\end{equation}
we can rewrite the equation of motion
\begin{eqnarray}
\frac{D\f^a}{\ud t} + 3H \dot\f^a + \gamma^{ab} V_{; b} = 0 ~. 
\end{eqnarray}
where we have $V_{;b} \equiv D V / d \f^b = \pd V/\pd \phi^b$, and
\[
\Gamma^a_{bc} = \frac{1}{2} \gamma^{ad}\left( \gamma_{db, c} + \gamma_{dc,b} - \gamma_{bc, d} \right)
\]

We define the inflaton velocity
\[
{\dot\f}_0^2 \equiv \gamma_{ab} {\dot\f}^a {\dot\f}^b ~. 
\]
The Friedmann equations gives
\begin{eqnarray}\label{fr_1}
3H^2\mpl^2 &=& \frac{1}{2} {\dot\f}_0^2 + V ~, \\ 
2\dot{H} \mpl^2 &=& - {\dot\f}_0^2 ~. \label{fr_2}
\end{eqnarray}

We can regard the composite field $\phi_0$ as the clock of multifield inflation. It is the classical field defined along the inflaton trajectory, and represents the length of the classical path.

\section{The Mode Equation}

\paragraph{Scalar perturbations:} To get a first order equation of motion for the perturbations $\delta \phi_I$, we need to consider the expansion of the action~\eqref{eqn:scalaraction} to second-order as well as the first-order scalar perturbations to the FLRW metric, given by
\begin{equation}
  ds^2 = -  \left(1 + 2 \Phi \right) dt^2 - 2 \, a^2 B_{,i} \, dt \,  dx^i + a^2 \left[ \left(1 - 2\Psi \right) \delta_{ij} - 2E_{,ij} \right] dx^i dx^j \, .
  \label{eqn:XXX}
\end{equation}
We can vary the first-order expanded action $\delta S_\phi$ with respect to the perturbations $\delta \phi_i$ to get the first-order equation of motion for the free-field perturbations \lp{expression without specifying gauge}
\begin{equation}
  test
  \label{eqn:XXX}
\end{equation}
With the Einstein field equations we have XXX equations for XXX variables.  \lp{Clearly explain gauge}  We choose the gauge XXX with XXX, leaving \lp{expression for lapse, etc}
\begin{equation}
  test
  \label{eqn:XXX}
\end{equation}
Substituting THIS into THAT, Fourier-transforming the scalar perturbations to $\delta \phi_I(\bk)$, and changing $t$ to $\alpha$ gives the mode equation as

Let $Q^a_\bk$ denote the scalar field perturbation in flat gauge, the mode equation reads \cite{Sasaki:1995aw}
\begin{equation}
\frac{D^2 Q_\bk^a}{\ud t^2} + 3H \frac{D Q_\bk^a}{\ud t} + \frac{k^2}{a^2} Q^a_\bk + {C^a}_b Q^b_\bk = 0
\end{equation}
\begin{equation}
{C^a}_b = \gamma^{ac}V_{;c;b} - {\dot\f_0}^2 {R^a}_{bcd} T^c T^d + 2\epsilon \frac{H}{\dot\f_0} (T^a V_{;b} + T^b V_{;a}) + 2\epsilon(3-\epsilon) H^2 T^a T_b
\end{equation}
Here $T^a$ is the unit tangent vector along the inflaton trajectory
$T^a \equiv \dot{\f}^a/\dot{\f}_0$ and
\begin{equation}
  \epsilon\equiv-\frac{\dot H}{H^2}=-\frac{1}{H}\frac{dH}{d\alpha}=\frac{1}{2}\sum_i \frac{d\phi_i}{d\alpha} \frac{d\phi_i}{d\alpha}.
  \label{eqn:XXX}
\end{equation}

\section{The Canonical Case (what's actually coded in )}

The equations simplify significantly in the canonical case. The background equation becomes 
\begin{equation}
\ddot{\phi}_I + 3H \dot{\phi}_I + \frac{\pd V}{\pd \phi_I} = 0  
\end{equation}
\begin{equation}\label{eq_H}
3H^2 \mpl^2 = \sum_I \dot{\phi}_I^2 + V(\phi_I)
\end{equation}
\begin{equation}
2\dot{H} \mpl^2 = -\sum_I \dot{\phi}_I^2
\end{equation}

For the mode equation, we have
\begin{equation}
\frac{d^2 Q_I}{\ud t^2} + 3H \frac{d Q_I}{\ud t} + \frac{k^2}{a^2} Q_I + \sum_J C_{IJ} Q_J = 0
\end{equation}
where the mass matrix is
\begin{equation}
C_{IJ} = \pd_I\pd_J V + \frac{1}{H \mpl^2} \left( \dot{\phi}_I \pd_J V + \dot{\phi}_J \pd_I V \right)
+ (3-\epsilon) \frac{\dot{\phi}_I\dot{\phi}_J}{\mpl^2}
\label{eqn:massmatrix}
\end{equation}

In practice, it is more convenient to evolve the equation in e-folds. We define $\alpha \equiv \ln a(t)$ and set $\mpl = 1$. 
Then the equations in variable $\alpha$ are 
\begin{equation}\label{eq_bk}
\frac{\ud^2 \phi_I}{\ud \alpha^2 } + (3 -\epsilon) \frac{\ud \phi_I}{\ud \alpha} + \frac{1}{H^2}\frac{\pd V}{\pd \phi_I} = 0  
\end{equation}
\begin{equation}
\frac{d^2 Q_I}{\ud \alpha^2} + (3-\epsilon) \frac{d Q_I}{\ud \alpha} + \frac{k^2}{a^2H^2} Q_I + \sum_J \tilC_{IJ} Q_J = 0
\end{equation}
\begin{equation}\label{cij}
\tilC_{IJ} = \frac{\pd_I\pd_J V}{H^2} + \frac{1}{H^2} \left(\frac{\ud \phi_I}{\ud \alpha} \pd_J V  +  \frac{\ud \phi_J}{\ud \alpha} \pd_I V \right)
+ (3-\epsilon) \frac{\ud \phi_I}{\ud \alpha} \frac{\ud \phi_J}{\ud \alpha}
\end{equation}

\section{Numerical Schemes}

\subsection{The Background Equation}

Solving the background equation is relatively straightforward. One start with Eq.(\ref{eq_bk}), and supplement it with Eq.(\ref{eq_H}). The two equations we solve simultaneously are 
\begin{equation}
H^2 = \frac{V(\phi_I)}{3 - \frac{1}{2}\sum_I \left(\frac{\ud \phi_I}{\ud \alpha}\right)^2} ~,
\end{equation}
\begin{equation}
\frac{\ud^2 \phi_I}{\ud \alpha^2 } + (3 -\epsilon) \frac{\ud \phi_I}{\ud \alpha} + \frac{1}{H^2}\frac{\pd V}{\pd \phi_I} = 0  ~.
\end{equation}
The initial $\phi_I$ and $\ud \phi_I/\ud \alpha$ have to be specified at the beginning. And the integration ends when $\epsilon = 1$ or when $\phi_I$'s satisfy certain conditions that signal the end of inflation. 

\subsection{The Perturbation Modes}

To solve the perturbation equations, it is usually convenient to work with the variable $u_I \equiv a Q_I$, so that specifying initial conditions is straightforward. 

The mode equation for $u_I$ is 
\begin{equation}
\frac{d^2 u_I}{\ud \alpha^2} + (1-\epsilon) \frac{d u_I}{\ud \alpha} + \left(\frac{k^2}{a^2H^2} - 2 + \epsilon \right) u_I + \sum_J \tilC_{IJ} u_J = 0
\label{eqn:ptbmodeeqn}
\end{equation}
with $\tilC_{IJ}$ given in Eq.(\ref{cij}).

Since the mass matrix $m^2_{IJ} \equiv \pd_I\pd_J V$ is not necessarily diagonal, the perturbation equations~\eqref{eqn:ptbmodeeqn} mix the annihilation operators for all of the fields \cite{Salopek:1988qh}. We therefore need to expand each perturbation mode $u_I(\bk)$ and $u_I^\dagger(\bk)$ using $N$ harmonic oscillators $a_J(\bk)$,
\begin{equation}
u_I(\bk, \alpha) = \sum_J \psi_{IJ}(\bk, \alpha) a_J(\bk) ~,
\end{equation}
and
\begin{equation}
u_I^\dagger(\bk, \alpha) = \sum_J \psi_{IJ}^*(\bk, \alpha) a_J^\dagger(\bk) ~,
\end{equation}
with \lp{Double-check factors of $\pi$.}
\begin{equation}
[ a_J(\bk), a^\dagger_I(\bk')] = (2\pi)^3 \delta_{IJ} \delta^{(3)}(\bk - \bk').
\end{equation}

The mode matrix $\psi_{IJ}$ evolves according to 
\begin{equation}
{\color{blue}
\frac{d^2 \psi_{IJ}}{\ud \alpha^2} + (1-\epsilon) \frac{d \psi_{IJ}}{\ud \alpha} + \left(\frac{k^2}{a^2H^2} - 2 + \epsilon \right) \psi_{IJ} + \sum_L \tilC_{IL} \psi_{LJ} = 0 ~. \label{mmeq} }
\end{equation}

For modes deep in the horizon, $\psi_{IJ}$ obeys the free wave equation (in conformal time $\tau$)
\begin{equation}
\frac{\ud^2 \psi_{IJ}}{\ud \tau^2} + k^2 \psi_{IJ} = 0 ~. 
\end{equation}
We assume that the mode matrix is initially diagonal at $\tau = -\infty$, 
\begin{eqnarray}
\psi_{IJ} = \frac{1}{\sqrt{2k}}\left( C_1 e^{ik\tau} + C_2 e^{-ik\tau} \right) \delta_{IJ}
\end{eqnarray}

Translating to e-fold time, the initial conditions can be set by
\begin{equation}
{\color{blue}
 \psi_{IJ}\Big|_{\alpha=0} = \frac{1}{\sqrt{2k}}\, (C_1 + C_2)\, \delta_{IJ} ~, 
\quad\quad 
\frac{\ud \psi_{IJ}}{\ud \alpha}\Big|_{\alpha=0} = \frac{i}{aH}\sqrt{\frac{k}{2}}\,(C_1 - C_2)\,\delta_{IJ} ~. }
\end{equation}
The convention coded in is $C_1 = 0$, $C_2 = 1$, but can be easily generalized to non Bunch-Davis cases.

{\color{blue} Eq.(\ref{mmeq}) is the main equation to evolve to get the power spectrum. }

Although $u_I$'s are convenient to use for short wavelength modes, they grow with the scale factor after the modes exit the horizon. So once the mode is outside the horizon, the $Q_I$ variables are easier to deal with numerically. The code can perform a switch from $u_I$ to $Q_I$ by matching boundary conditions at certain time $\alpha^*$, i.e.
\begin{equation}
u_I\Big|_{\alpha^*} = e^{\alpha^*} Q_I \Big|_{\alpha^*} ~, \quad \quad 
\frac{\ud u_I}{\ud \alpha}\Big|_{\alpha^*} = e^{\alpha^*} \left(Q_I + \frac{\ud Q_I}{\ud \alpha}\right)\Big|_{\alpha^*}
\end{equation}


\section{The Power Spectrum}

After solving the background and perturbation equations numerically. We have the trajectory of the inflaton field 
\[
(\phi_1(\alpha), \phi_2(\alpha), \dots, \phi_N(\alpha) ) ~,
\]
plus the perturbations
\[
(Q_1, Q_2, \dots, Q_N) ~. 
\]

Unlike the single field case, now the power spectrum takes the matrix form. Suppose inflation ends at e-fold time $\alpha_e$, then
\begin{equation}
{\color{blue}
P_Q^{IJ} (k) = \frac{k^3}{2\pi^2} \sum_L Q_{IL}Q^*_{JL} =
\frac{k^3}{2\pi^2} \sum_L \frac{\psi_{IL}\psi^*_{JL}}{a^2}
}
\end{equation}

The adiabatic perturbation spectrum is the projection along the inflaton direction, define a field vector
\begin{equation}
\omega_\zeta = \left( \frac{\dot{\phi_1}}{\dot{\phi_0}},\, \frac{\dot{\phi_2}}{\dot{\phi_0}},\, \dots,\, \frac{\dot{\phi_N}}{\dot{\phi_0}} \right),
\end{equation}
where
\begin{equation}
  \zeta \equiv \Psi - H \frac{\delta \rho}{\dot \rho} = -H \frac{ \sum_I \delta \phi_I}{\sum_J \dot \phi_J \dot \phi_J} = -\frac{H}{\dot \phi_0}\sum_I \omega_\zeta^I \delta \phi_I
  \label{eqn:XXX}
\end{equation}
is the curvature perturbation in flat gauge ($\Psi=0$) \lp{How do we argue that $\partial(\delta \phi_I)=0$?}.  The adiabatic, curvature power spectrum is then given by
\begin{equation}
{\color{blue}
  P_\zeta(k) = \frac{H^2}{\dot{\phi_0}^2} \sum_{IJ} \omega_\zeta^I \omega_\zeta^J P_Q^{IJ} = \frac{1}{2 \epsilon} \sum_{IJ} \omega_\zeta^I \omega_\zeta^J P_Q^{IJ}
}
\end{equation}

Note that $\omega_\zeta^I$ is just a projection of the perturbation modes to the inflaton direction at the end of inflation. However, each mode $Q_I$ evolves non-trivially, as they are coupled to each other. The coupling is turned on each time the classical trajectory of inflaton bends. In other words, we have already included the conversion between entropic and adiabatic modes during the course of inflation by evolving each mode $Q_I$ towards the end of inflation. 

Similarly, one can project out iso-curvature spectrum.  Given the adiabatic unit vector $\omega_\zeta$ we can construct an isocurvature basis spanned by $N-1$ vectors $s_i$ in terms of the $N$ field space unit vectors $\hat \phi_i$ as
\begin{equation}
  s_i = \frac{\hat \phi_i - (\hat \phi_i \cdot \omega_\zeta) \omega_\zeta}{|\hat \phi_i - (\hat \phi_i \cdot \omega_\zeta) \omega_\zeta|}.
  \label{eqn:XXX}
\end{equation}
There are then $N-1$ iso-curvature spectra given by
\begin{equation}
  P_{\alpha}^S (k) = \sum_{IJ} s_\alpha^I s_\alpha^J P_Q^{IJ}.
\end{equation}
This gives 

\section{Orphans}

Mainly orphaned statements from the original draft of the ICs+preds paper

   The \textbf{comoving entropy perturbation spectrum} $\Pent$ is...

   The \textbf{non-adiabatic pressure perturbation spectrum} $\Ppnad$ is ...

    \lp{ Note the issue with double-precision calculations of $\mathcal P_\mathrm{Pnad}$.}

However, we can see that a different quantity source the change in $\calR$.  Differentiating Eq.~\eqref{eqn:rginvariant} gives
\begin{equation}
  \dot \calR = -\frac{H}{\df^2} \delta P_\mathrm{nad}
  \label{eqn:XXX}
\end{equation}
where the non-adiabatic pressure perturbation $\delta P_\mathrm{nad}$ is the difference between the total pressure perturbation $\delta P$ and the adiabatic pressure perturbation $\delta P_\mathrm{ad}$ :
\begin{equation}
  \delta P_\mathrm{nad} \equiv \delta P - c_s^2 \delta \rho
  \label{eqn:XXX}
\end{equation}
is  given in terms of the total pressure perturbation
\begin{equation}
  \delta P = test \, ;
  \label{eqn:XXX}
\end{equation}
the total density perturbation
\begin{equation}
  test
  \label{eqn:XXX}
\end{equation}
and the speed of sound $c_s^2 = \dot {\bar P} / \dot {\bar \rho}$.

\section{Observables}

\begin{table}
  \centering
  \begin{tabular}{  c  c c  c }
    \hline
    Observable &  Name & Description & Reference \\
    \hline
    $A_s$ \dotfill             &   Scalar amplitude & $\PR(k_*)$  & Eq.~\eqref{eqn:pad}  \\
    $A_\mathrm{iso}$  \dotfill &   Isocurvature amplitude & $\PS(k_*)$ & Eq.~\eqref{eqn:piso}  \\
   $A_\mathrm{Cross}$ \dotfill &   Cross-spectra ampl.& $\Cross(k_*)$ & Eq.~\eqref{eqn:cross}  \\
    $A_\mathrm{ent}$  \dotfill &   Entropy amplitude& $\Pent(k_*)$ & Eq.~\eqref{eqn:pent}  \\
    $A_\mathrm{Pnad}$ \dotfill &   Non-adiab. pressure ampl.& $\Ppnad(k_*)$ & Eq.~\eqref{eqn:ppnad}  \\
    \hline

    $n_s$  \dotfill            &    Spectral tilt & $\mathcal D_* \log \PR+1$& Eq. XXX  \\
    $n_T$  \dotfill            &    Tensor spectral tilt & $\mathcal D_* \log \Pten$& Eq. XXX  \\
    $n_\mathrm{iso}$  \dotfill &    Isocurv. tilt & $\mathcal D_* \log \PS$& Eq. XXX  \\
    $n_\mathrm{ent}$  \dotfill &    Entropy tilt & $\mathcal D_* \log \Pent$& Eq. XXX \\
    $n_\mathrm{Pnad}$ \dotfill &    Non-adiab. pressure tilt & $\mathcal D_* \log \Ppnad$& Eq. XXX  \\
    \hline

    $\alpha_s$  \dotfill     &  Spectral running & $\mathcal D_*^2 \log \PR$ & Eq. XXX  \\
    $r$    \dotfill          &  Tensor-to-scalar ampl. & $\Pten(k_*)/\PR(k_*)$ & Eq. XXX  \\
    $\Theta$ \dotfill            &  Bundle width?  & \lp{Jonny}  & Eq. XXX  \\
    $\cos \Delta$ \dotfill            &  $\omega-s$ correl. angle & $\Cross/\sqrt{\mathcal P_{\mathcal R} \mathcal P_{\mathcal S}}$ at $k_*$   & Eq. XXX  \\
    \hline
  \end{tabular}

  \caption{Typical observables; $k_*=0.002 \, \mathrm{Mpc}^{-1}$ is the pivot scale.    $\mathcal D_* \equiv d/d \log k$ evaluated at $k=k_*$.}
  \label{table:observables}
\end{table}


\section{Lit Review}


\begin{itemize}

  \item  Langlois paper: curvature/mass basis \cite{Gao:2013ota}

  \item  Japanese group; analytical turning point \cite{Noumi:2013cfa}

  \item Effectively single-field analytics \cite{Noumi:2012vr}

  \item Integrating out massive modes to get low-energy effective field theory.  Studies result of this breakdown during turn; analytical \cite{Gao:2012uq}

  \item Analytic computation; matches EFT argument \cite{Pi:2012gf}

  \item EFT analytics.  Shows that the EFT remains viable until the rate of change of the angular velocity during the term becomes on the order of the most massive masses.  Valid for a large mass hierarchy, but may break down as the masses become approximately equal.  \cite{Cespedes:2012hu}

  \item \emph{Relevant}: Shows that typical $\delta N$ expansion to first order isn't enough in the case of hybrid inflation with mildly tachyonic directions where the isocurvature modes source the curvature perturbations.  Analytical and numerical techniques in double-quadratic and double-quartic turns. \cite{Avgoustidis:2011em}

  \item \emph{Relevant}: Numerical calculations using Pyflation of curvature and isocurvature power spectra in double-quadratic, hybrid, product exponential \cite{Huston:2011fr}

  \item \emph{Relevant}: Numerical calculations with different mass heirarchies.  Calculates damping of entropy mades after horizon exit in variety of models and categorizes SR quantities by what effect they have on the power spectrum. \cite{Peterson:2010np}

  \item \emph{Relevant}: Isocurvature perturbation stuff \cite{Huston:2011fr,Gordon:2000hv}

  \item Analytical turning \cite{Achucarro:2010da}

  \item Quasi-single field turning trajectories.  Analytic, in-in \cite{Chen:2009zp,Chen:2009we}

\end{itemize}

%%%%%%%%%%%%%%%%%%%%%%%%%%%%%%%%%%%%%%%%%%%%%%%%%%%%5
\section{Notes on numerical checks}

\subsection{Integrator accuracy}

Requires increase in accuracy whenever $\epsilon \sim 1$ since $Q_{ij}$ grows quasi-exponentially

\subsection{$P(k)$ observables}

Errors associated with approximating the full numerical $P(k)$ with finite difference.

It is often assumed that in the range of modes that are relevant for CMB analysis the power spectrum is well-approximated in log-space as
\begin{equation}
  \log \left( \frac{\mathcal P}{A_s} \right) = \left[ n_s - 1 + \frac{1}{2} \alpha_s \log \left( \frac{k}{k_*} \right) \right]\log \left( \frac{k}{k_*} \right) \, .
  \label{eqn:runningpk}
\end{equation}
We estimate $n_s$ using first-order central finite difference between two amplitudes $\calP_1$ and $\calP_2$ evaluated at the scales $k_1 = k_* \exp (-\delta)$ and $k_2= k_* \exp \delta$,
\begin{equation}
  n_s \approx 1 + \frac{1}{2\delta}\log \left( \frac{ \calP_2 }{ \calP_1} \right),
  \label{eqn:nsapprox}
\end{equation}
which is an $\calO(\delta^2)$ accurate estimate for $d\log \calP/d \log k$ (and consequently $n_s$ if Eq.~\eqref{eqn:runningpk} holds).  Similarly, we use second-order central finite difference to estimate the running of the spectral index $\alpha_s$ as
\begin{equation}
  \alpha_s \approx \frac{1}{\delta^2} \log \left( \frac{\calP_2 \calP_1}{A_s} \right).
  \label{eqn:alphaapprox}
\end{equation}

However, the power spectrum for many background trajectories is not well-approximated by Eq.~\eqref{eqn:runningpk}.  Define the deviation from Eq.~\eqref{eqn:runningpk} at a scale $k_i$ by $\Delta_i$  and assume that numerical error at the pivot scale is negligible \lp{expand}. While Eqs.~\eqref{eqn:nsapprox}--\eqref{eqn:alphaapprox} are $\calO (\delta^2)$ approximations to the first-- and second-order derivatives of $\log \calP$, respectively, the error $\mathbb E$ between these derivatives and $n_s$ or $\alpha_s$ depends on how well the ansatz fits the real spectrum.  This can be approximated by
\begin{eqnarray}
  \mathbb E (n_s) = \Delta/2 \delta \, , \\
  \mathbb E (\alpha_s) = \Delta/\delta^2 \, ,
  \label{eqn:XXX}
\end{eqnarray}
where $\Delta \equiv \Delta_2 - \Delta_1$.  The estimates for $n_s$ and $\alpha_s$ are then dependent on the step-size $\delta$ used in the finite-difference calculation, in particular requiring $\delta^2 \gg \Delta$, where $\Delta$ is obviously hard to estimate \emph{a priori} for general models and all allowed initial conditions.  One would ideally like $\delta \ll 1$ to minimize the $\calO (\delta^2)$ finite-difference error.

Note that there is only a marginal improvement in the finite-difference estimates when using a 5-point stencil over the central difference approximations and it requires two more calls to the evolve routine.


For certain trajectories the second-order finite difference breaks down for the calculation of $\alpha_s$ when it should be approximately zero.  Can fix this by increasing the distance between evaluated points $h \equiv \Delta \ln k$.


\subsection{$\delta P_\mathrm{nad}$ calculation}

Importantly, when calculating $\delta P_\mathrm{nad} \equiv \delta P - c_s^2 \delta \rho$, $\delta P$ remains of the same order of magnitude throughout the evolution, but during the approach to the adiabatic limit the adiabatic pressure perturbation increases to this level.  Consequently, we run into roundoff error problems and a loss of significance in this calculation.  This is fixed to a large extent by calculating the projected isocurvature power spectrum since the amplitude of the isocurvature modes should decrease as we approach the adiabatic limit, making any differences calculated $X-Y$ of the same order.  We can therefore trust this comparison to a much higher degree of accuracy.

\subsection{Other}

\begin{itemize}

  \item Isocurvature spurious projection onto the adiabatic direction can be dominant contribution whenthe isocurvature is small.  Optional fix in modpk\_potential.f90 with renormalize\_remove\_smallest(omega\_z).


\end{itemize}


\section{Gram-Schmidt orthogonalization}
\label{sect:gsorthog}

\lp{Explain the Gram-Schmidt better.}
  Since Eqs.XXX are expressed in terms of the fields $\phi_I$, we define the field-space orthonormal basis as $\phi_I = \phi_I/||\phi_I||$ and order them at each time step so that the first field-basis vector is 
\begin{equation}
  \phi_0 \equiv \mathrm{max} [\phi_i \cdot \omega ]
  \label{eqn:XXX}
\end{equation}
which is the one most-aligned with $\omega$.  We do this since $\phi_0$ might be almost identical to $\omega$, for instance when inflation is being effectively driven by only one field.  The first isocurvature direction\footnote{This differs from setting the first isocurvature direction in the direction proportional to $\ddot \phi$, the direction of curvature, which according to Wikipedia is called the Frenet-Serret basis.}
is then the projection of the first field-direction $\phi_1$ orthogonal to $\phi_0$,
\begin{equation}
  s_1 = \frac{\phi_1 - ( \phi_1 \cdot \omega ) \omega}{1-(\phi_1 \cdot \omega)^2},
  \label{eqn:s1_gs}
\end{equation}
and the remaining $N_f-2$ isocurvature directions are built from $s_1$ by
\begin{equation}
   s_K = \frac{  \phi_K - \sum_{J=1}^{K-1} (  \phi_K \cdot  s_J )  s_J}{1-\sum_{J=1}^{K-1} (  \phi_K \cdot  s_J )^2} \qquad \mathrm{for} \qquad K>1 \, .
  \label{eqn:sk_gs}
\end{equation}


%    \lp{Be careful with $s_J$, which is the $J$-th vector and $\omega_J$ which is the $J$-th component} The $\phi_{\mathrm{iso},K}^\prime$ can be determined iteratively by differentiating Eqs.~\eqref{eqn:s1_gs},~\eqref{eqn:sk_gs},~and~\eqref{eqn:phiad} to get
%    \begin{equation}
%       s_K^\prime = -\frac{1}{||s_K||} \left[ \sum_{J=1}^{K-1} \left( ( \phi_K \cdot  s_J^\prime)  s_J + ( \phi_K \cdot  s_J)  s_J^\prime \right) +  s_K ||s_K||^\prime \right]
%      \label{eqn:XXX}
%    \end{equation}
%    where the normalization factor and its derivative is
%    \begin{eqnarray}
%      ||s_K|| &\equiv& 1-\sum_{J=1}^{K-1}(  \phi_K \cdot  s_J )^2 \, , \\
%      ||s_K||^\prime &\equiv& - 2 \sum_{J=1}^{K-1} \left(  \phi_K \cdot  s_J \right) \left( \phi_K \cdot  s_J^\prime \right)
%      \label{eqn:XXX}
%    \end{eqnarray}
%    for $K>1$.  \lp{To get $ s_1^\prime$ use $\omega$.}



\bibliographystyle{plain}

\bibliography{references}


\end{document}
