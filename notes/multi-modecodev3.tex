\documentclass[11pt]{article}

\usepackage{amssymb}
\usepackage{amsmath}
\usepackage{fullpage}
\usepackage{color}

\newcommand{\mpl}{m_{\mathrm{pl}}}
\newcommand{\ud}{\mathrm{d}}
\newcommand{\pd}{\partial}
\def\calD{{\cal D}}
\def\bk{{\mathbf{k}}}
\def\f {{\phi}}

\setlength{\parskip}{1.5ex}
\setlength{\parindent}{0pt}
\def\tilC{{\tilde C}}
\def\calR{{\cal R}}

%Author comments.
%===============
\newcommand{\re}[1]{\textcolor{green}{[{\bf RE}: #1]}}
\newcommand{\lp}[1]{\textcolor{red}{[{\bf LP}: #1]}}
\newcommand{\jf}[1]{\textcolor{red}{[{\bf JF}: #1]}}
\newcommand{\jx}[1]{\textcolor{red}{[{\bf JX}: #1]}}
\newcommand{\hp}[1]{\textcolor{red}{[{\bf HP}: #1]}}

\title{Equations for Multifield Modecode v3.0}
\author{Jiajun Xu}
\date{\today}

\begin{document}

\maketitle

We start with the action for $N$ scalar fields $\phi^a$, $(a=1,2, \dots, N)$. 
\begin{equation}
S = \int \ud^4 x \sqrt{-g} \left[ -\frac{1}{2} \gamma_{ab} \pd_\mu \phi^a \pd^\mu \phi^b - V(\f^a) \right]
\end{equation}
We will first write down equations for the general metric $\gamma_{ab}$, and then reduce to the canonical case $\gamma_{ab} =\delta_{ab}$. 

The current stage of Multifield Modecode only deals with canonical fields. 

\section{The Classical Background}

\begin{equation}\label{bg_phi}
\ddot{\f^a} + \Gamma^a_{bc} \dot{\f^b} \dot{\f^c} + 3H \dot\f^a + \gamma^{ab} \frac{\pd V}{\pd \f^b} = 0  ~.
\end{equation}
Introducing the covariant derivative \lp{Need to clarify what this structure is... differential?}
\begin{equation}
D \f^a = \ud \f^a + \Gamma^a_{bc}\dot{\f^b}\ud\f^c ~,
\end{equation}
we can rewrite the equation of motion
\begin{eqnarray}
\frac{D\f^a}{\ud t} + 3H \dot\f^a + \gamma^{ab} V_{; b} = 0 ~. 
\end{eqnarray}
where we have $V_{;b} \equiv D V / d \f^b = \pd V/\pd \phi^b$, and
\[
\Gamma^a_{bc} = \frac{1}{2} \gamma^{ad}\left( \gamma_{db, c} + \gamma_{dc,b} - \gamma_{bc, d} \right)
\]

We define the inflaton velocity
\[
{\dot\f}_0^2 \equiv \gamma_{ab} {\dot\f}^a {\dot\f}^b ~. 
\]
The Friedmann equations gives
\begin{eqnarray}\label{fr_1}
3H^2\mpl^2 &=& \frac{1}{2} {\dot\f}_0^2 + V ~, \\ 
2\dot{H} \mpl^2 &=& - {\dot\f}_0^2 ~. \label{fr_2}
\end{eqnarray}

We can regard the composite field $\phi_0$ as the clock of multifield inflation. It is the classical field defined along the 
inflaton trajectory, and represents the length of the classical path. 

\section{The Mode Equation}

Let $Q^a_\bk$ denote the scalar field perturbation in flat gauge, the mode equation reads \cite{Sasaki:1995aw}
\begin{equation}
\frac{D^2 Q_\bk^a}{\ud t^2} + 3H \frac{D Q_\bk^a}{\ud t} + \frac{k^2}{a^2} Q^a_\bk + {C^a}_b Q^b_\bk = 0
\end{equation}
\begin{equation}
{C^a}_b = \gamma^{ac}V_{;c;b} - {\dot\f_0}^2 {R^a}_{bcd} T^c T^d + 2\epsilon \frac{H}{\dot\f_0} (T^a V_{;b} + T^b V_{;a}) + 2\epsilon(3-\epsilon) H^2 T^a T_b
\end{equation}
Here $T^a$ is the unit tangent vector along the inflaton trajectory
$T^a \equiv \dot{\f}^a/\dot{\f}_0$ and
\begin{equation}
  \epsilon\equiv-\frac{\dot H}{H^2}=-\frac{1}{H}\frac{dH}{d\alpha}=\frac{1}{2}\sum_i \frac{d\phi_i}{d\alpha} \frac{d\phi_i}{d\alpha}.
  \label{eqn:XXX}
\end{equation}

\section{The Canonical Case (what's actually coded in )}

The equations simplify significantly in the canonical case. The background equation becomes 
\begin{equation}
\ddot{\phi}_I + 3H \dot{\phi}_I + \frac{\pd V}{\pd \phi_I} = 0  
\end{equation}
\begin{equation}\label{eq_H}
3H^2 \mpl^2 = \sum_I \dot{\phi}_I^2 + V(\phi_I)
\end{equation}
\begin{equation}
2\dot{H} \mpl^2 = -\sum_I \dot{\phi}_I^2
\end{equation}

For the mode equation, we have
\begin{equation}
\frac{d^2 Q_I}{\ud t^2} + 3H \frac{d Q_I}{\ud t} + \frac{k^2}{a^2} Q_I + \sum_J C_{IJ} Q_J = 0
\end{equation}
where the mass matrix is
\begin{equation}
C_{IJ} = \pd_I\pd_J V + \frac{1}{H \mpl^2} \left( \dot{\phi}_I \pd_J V + \dot{\phi}_J \pd_I V \right)
+ (3-\epsilon) \frac{\dot{\phi}_I\dot{\phi}_J}{\mpl^2}
\label{eqn:massmatrix}
\end{equation}

In practice, it is more convenient to evolve the equation in e-folds. We define $\alpha \equiv \ln a(t)$ and set $\mpl = 1$. 
Then the equations in variable $\alpha$ are 
\begin{equation}\label{eq_bk}
\frac{\ud^2 \phi_I}{\ud \alpha^2 } + (3 -\epsilon) \frac{\ud \phi_I}{\ud \alpha} + \frac{1}{H^2}\frac{\pd V}{\pd \phi_I} = 0  
\end{equation}
\begin{equation}
\frac{d^2 Q_I}{\ud \alpha^2} + (3-\epsilon) \frac{d Q_I}{\ud \alpha} + \frac{k^2}{a^2H^2} Q_I + \sum_J \tilC_{IJ} Q_J = 0
\end{equation}
\begin{equation}\label{cij}
\tilC_{IJ} = \frac{\pd_I\pd_J V}{H^2} + \frac{1}{H^2} \left(\frac{\ud \phi_I}{\ud \alpha} \pd_J V  +  \frac{\ud \phi_J}{\ud \alpha} \pd_I V \right)
+ (3-\epsilon) \frac{\ud \phi_I}{\ud \alpha} \frac{\ud \phi_J}{\ud \alpha}
\end{equation}

\section{Numerical Schemes}

\subsection{The Background Equation}

Solving the background equation is relatively straightforward. One start with Eq.(\ref{eq_bk}), and supplement it with Eq.(\ref{eq_H}). The two equations we solve simultaneously are 
\begin{equation}
H^2 = \frac{V(\phi_I)}{3 - \frac{1}{2}\sum_I \left(\frac{\ud \phi_I}{\ud \alpha}\right)^2} ~,
\end{equation}
\begin{equation}
\frac{\ud^2 \phi_I}{\ud \alpha^2 } + (3 -\epsilon) \frac{\ud \phi_I}{\ud \alpha} + \frac{1}{H^2}\frac{\pd V}{\pd \phi_I} = 0  ~.
\end{equation}
The initial $\phi_I$ and $\ud \phi_I/\ud \alpha$ have to be specified at the beginning. And the integration ends when $\epsilon = 1$ or when $\phi_I$'s satisfy certain conditions that signal the end of inflation. 

\subsection{The Perturbation Modes}

To solve the perturbation equations, it is usually convenient to work with the variable $u_I \equiv a Q_I$, so that specifying initial conditions is straightforward. 

The mode equation for $u_I$ is 
\begin{equation}
\frac{d^2 u_I}{\ud \alpha^2} + (1-\epsilon) \frac{d u_I}{\ud \alpha} + \left(\frac{k^2}{a^2H^2} - 2 + \epsilon \right) u_I + \sum_J \tilC_{IJ} u_J = 0
\label{eqn:ptbmodeeqn}
\end{equation}
with $\tilC_{IJ}$ given in Eq.(\ref{cij}).

Since the mass matrix $m^2_{IJ} \equiv \pd_I\pd_J V$ is not necessarily diagonal, the perturbation equations~\eqref{eqn:ptbmodeeqn} mix the annihilation operators for all of the fields \cite{Salopek:1988qh}. We therefore need to expand each perturbation mode $u_I(\bk)$ and $u_I^\dagger(\bk)$ using $N$ harmonic oscillators $a_J(\bk)$,
\begin{equation}
u_I(\bk, \alpha) = \sum_J \psi_{IJ}(\bk, \alpha) a_J(\bk) ~,
\end{equation}
and
\begin{equation}
u_I^\dagger(\bk, \alpha) = \sum_J \psi_{IJ}^*(\bk, \alpha) a_J^\dagger(\bk) ~,
\end{equation}
with \lp{Double-check factors of $\pi$.}
\begin{equation}
[ a_J(\bk), a^\dagger_I(\bk')] = (2\pi)^3 \delta_{IJ} \delta^{(3)}(\bk - \bk').
\end{equation}

The mode matrix $\psi_{IJ}$ evolves according to 
\begin{equation}
{\color{blue}
\frac{d^2 \psi_{IJ}}{\ud \alpha^2} + (1-\epsilon) \frac{d \psi_{IJ}}{\ud \alpha} + \left(\frac{k^2}{a^2H^2} - 2 + \epsilon \right) \psi_{IJ} + \sum_L \tilC_{IL} \psi_{LJ} = 0 ~. \label{mmeq} }
\end{equation}

For modes deep in the horizon, $\psi_{IJ}$ obeys the free wave equation (in conformal time $\tau$)
\begin{equation}
\frac{\ud^2 \psi_{IJ}}{\ud \tau^2} + k^2 \psi_{IJ} = 0 ~. 
\end{equation}
We assume that the mode matrix is initially diagonal at $\tau = -\infty$, 
\begin{eqnarray}
\psi_{IJ} = \frac{1}{\sqrt{2k}}\left( C_1 e^{ik\tau} + C_2 e^{-ik\tau} \right) \delta_{IJ}
\end{eqnarray}

Translating to e-fold time, the initial conditions can be set by
\begin{equation}
{\color{blue}
 \psi_{IJ}\Big|_{\alpha=0} = \frac{1}{\sqrt{2k}}\, (C_1 + C_2)\, \delta_{IJ} ~, 
\quad\quad 
\frac{\ud \psi_{IJ}}{\ud \alpha}\Big|_{\alpha=0} = \frac{i}{aH}\sqrt{\frac{k}{2}}\,(C_1 - C_2)\,\delta_{IJ} ~. }
\end{equation}
The convention coded in is $C_1 = 0$, $C_2 = 1$, but can be easily generalized to non Bunch-Davis cases. 

{\color{blue} Eq.(\ref{mmeq}) is the main equation to evolve to get the power spectrum. }

Although $u_I$'s are convenient to use for short wavelength modes, they grow with the scale factor after the modes exit the horizon. So once the mode is outside the horizon, the $Q_I$ variables are easier to deal with numerically. The code can perform a switch from $u_I$ to $Q_I$ by matching boundary conditions at certain time $\alpha^*$, i.e.
\begin{equation}
u_I\Big|_{\alpha^*} = e^{\alpha^*} Q_I \Big|_{\alpha^*} ~, \quad \quad 
\frac{\ud u_I}{\ud \alpha}\Big|_{\alpha^*} = e^{\alpha^*} \left(Q_I + \frac{\ud Q_I}{\ud \alpha}\right)\Big|_{\alpha^*}
\end{equation}


\section{The Power Spectrum}

After solving the background and perturbation equations numerically. We have the trajectory of the inflaton field 
\[
(\phi_1(\alpha), \phi_2(\alpha), \dots, \phi_N(\alpha) ) ~,
\]
plus the perturbations
\[
(Q_1, Q_2, \dots, Q_N) ~. 
\]

Unlike the single field case, now the power spectrum takes the matrix form. Suppose inflation ends at e-fold time $\alpha_e$, then
\begin{equation}
{\color{blue}
P_Q^{IJ} (k) = \frac{k^3}{2\pi^2} \sum_L Q_{IL}Q^*_{JL} =
\frac{k^3}{2\pi^2} \sum_L \frac{\psi_{IL}\psi^*_{JL}}{a^2}
}
\end{equation}

The adiabatic perturbation spectrum is the projection along the inflaton direction, define a field vector
\begin{equation}
\omega_\zeta = \left( \frac{\dot{\phi_1}}{\dot{\phi_0}},\, \frac{\dot{\phi_2}}{\dot{\phi_0}},\, \dots,\, \frac{\dot{\phi_N}}{\dot{\phi_0}} \right),
\end{equation}
where
\begin{equation}
  \zeta \equiv \Psi - H \frac{\delta \rho}{\dot \rho} = -H \frac{ \sum_I \delta \phi_I}{\sum_J \dot \phi_J \dot \phi_J} = -\frac{H}{\dot \phi_0}\sum_I \omega_\zeta^I \delta \phi_I
  \label{eqn:XXX}
\end{equation}
is the curvature perturbation in flat gauge ($\Psi=0$) \lp{How do we argue that $\partial(\delta \phi_I)=0$?}.  The adiabatic, curvature power spectrum is then given by
\begin{equation}
{\color{blue}
  P_\zeta(k) = \frac{H^2}{\dot{\phi_0}^2} \sum_{IJ} \omega_\zeta^I \omega_\zeta^J P_Q^{IJ} = \frac{1}{2 \epsilon} \sum_{IJ} \omega_\zeta^I \omega_\zeta^J P_Q^{IJ}
}
\end{equation}

Note that $\omega_\zeta^I$ is just a projection of the perturbation modes to the inflaton direction at the end of inflation. However, each mode $Q_I$ evolves non-trivially, as they are coupled to each other. The coupling is turned on each time the classical trajectory of inflaton bends. In other words, we have already included the conversion between entropic and adiabatic modes during the course of inflation by evolving each mode $Q_I$ towards the end of inflation. 

Similarly, one can project out iso-curvature spectrum.  Given the adiabatic unit vector $\omega_\zeta$ we can construct an isocurvature basis spanned by $N-1$ vectors $s_i$ in terms of the $N$ field space unit vectors $\hat \phi_i$ as
\begin{equation}
  s_i = \frac{\hat \phi_i - (\hat \phi_i \cdot \omega_\zeta) \omega_\zeta}{|\hat \phi_i - (\hat \phi_i \cdot \omega_\zeta) \omega_\zeta|}.
  \label{eqn:XXX}
\end{equation}
There are then $N-1$ iso-curvature spectra given by
\begin{equation}
  P_{\alpha}^S (k) = \sum_{IJ} s_\alpha^I s_\alpha^J P_Q^{IJ}.
\end{equation}
This gives 

\section{Lit Review}


\begin{itemize}

  \item  Langlois paper: curvature/mass basis \cite{Gao:2013ota}

  \item  Japanese group; analytical turning point \cite{Noumi:2013cfa}

  \item Effectively single-field analytics \cite{Noumi:2012vr}

  \item Integrating out massive modes to get low-energy effective field theory.  Studies result of this breakdown during turn; analytical \cite{Gao:2012uq}

  \item Analytic computation; matches EFT argument \cite{Pi:2012gf}

  \item EFT analytics.  Shows that the EFT remains viable until the rate of change of the angular velocity during the term becomes on the order of the most massive masses.  Valid for a large mass hierarchy, but may break down as the masses become approximately equal.  \cite{Cespedes:2012hu}

  \item \emph{Relevant}: Shows that typical $\delta N$ expansion to first order isn't enough in the case of hybrid inflation with mildly tachyonic directions where the isocurvature modes source the curvature perturbations.  Analytical and numerical techniques in double-quadratic and double-quartic turns. \cite{Avgoustidis:2011em}

  \item \emph{Relevant}: Numerical calculations using Pyflation of curvature and isocurvature power spectra in double-quadratic, hybrid, product exponential \cite{Huston:2011fr}

  \item \emph{Relevant}: Numerical calculations with different mass heirarchies.  Calculates damping of entropy mades after horizon exit in variety of models and categorizes SR quantities by what effect they have on the power spectrum. \cite{Peterson:2010np}

  \item \emph{Relevant}: Isocurvature perturbation stuff \cite{Huston:2011fr,Gordon:2000hv}

  \item Analytical turning \cite{Achucarro:2010da}

  \item Quasi-single field turning trajectories.  Analytic, in-in \cite{Chen:2009zp,Chen:2009we}

\end{itemize}



\bibliographystyle{plain}

\bibliography{references}


\end{document}
